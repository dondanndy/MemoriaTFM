\documentclass[12pt,a4paper,oneside]{article}
\usepackage[T1]{fontenc}
\usepackage[utf8]{inputenc}
\usepackage[spanish, es-tabla]{babel}
\usepackage{lmodern}
\usepackage{mathtools}
\usepackage{amsthm}
\usepackage{amsmath}
\usepackage{amsfonts}
\usepackage{amssymb}
\usepackage{makeidx}
\usepackage{graphicx}
\usepackage[table,xcdraw]{xcolor}
\usepackage{float}
\usepackage{hyperref}
\usepackage{wrapfig}
\usepackage{xfrac}

\usepackage[bottom=2.5cm, top = 2.5cm, right=2cm, left=2cm, bindingoffset=0.5cm]{geometry}

\usepackage{tikz}
\usetikzlibrary{arrows.meta, babel}
\usepackage{pgfplots}
\pgfplotsset{compat=newest}
\usetikzlibrary{arrows,shapes,positioning,calc,babel,pgfplots.groupplots, pgfplots.statistics}

\usepackage[font=small,justification=centering]{caption}
\usepackage{subcaption}
\usepackage{pgfkeys} % LATEX
\usepackage{pgffor}
\usepackage{adjustbox}
\usepackage{bm}
\usepackage{pdfpages}
\usepackage{siunitx}
\usepackage{multirow}
% \usepackage{multicol}
% \usepackage{afterpage}
\usepackage{xcolor}
\usepackage{soul}
% \usepackage{listings}
\usepackage{titlesec}
\usepackage[italicdiff]{physics}
\usepackage[Algoritmo]{algorithm}
\usepackage{algpseudocode}
\usepackage[onehalfspacing]{setspace}

%------------------------------------------------------------------------------
%               Encabezado y pie de página
%------------------------------------------------------------------------------

\usepackage{fancyhdr}

% aqui definimos el encabezado de las paginas pares e impares.
\lhead[]{\textsc{Simulación parelela de la \\ emisión de WiFi con trazado de rayos}}
\chead[]{}
\rhead[]{}
\lfoot[]{}
\cfoot[]{\thepage}
\rfoot[]{}
\renewcommand{\footrulewidth}{0pt}
\renewcommand{\headrulewidth}{0.5pt}


% aqui definimos el encabezado y pie de pagina de la pagina inicial de un capitulo.
\fancypagestyle{plain}{
	\fancyhead[L]{}
	\fancyhead[C]{}
	\fancyhead[R]{}
	\fancyfoot[L]{}
	\fancyfoot[C]{}
	\fancyfoot[R]{}
	\renewcommand{\headrulewidth}{0pt}
	\renewcommand{\footrulewidth}{0pt}
}

\pagestyle{fancy}

% \setlength{\parskip}{1em}

\begin{document}

%------------------------------------------------------------------------------
%                             Portada
%------------------------------------------------------------------------------

\includepdf[pages={1,2}]{portada.pdf}

\tableofcontents

\renewcommand{\thetable}{\thesection.\arabic{table}}
\renewcommand{\thefigure}{\thesection.\arabic{figure}}
\renewcommand{\theequation}{\thesection.\arabic{equation}}

%------------------------------------------------------------------------------
%                             Objetivos
%------------------------------------------------------------------------------

% \newpage
% \section{Objetivos}

% \input{tex/objetivos.tex}

\setlength{\parskip}{1em}

%------------------------------------------------------------------------------
%                             Introduccion
%------------------------------------------------------------------------------
\setcounter{table}{0}
\setcounter{equation}{0}
\setcounter{figure}{0}
\setcounter{page}{1}

\newpage
\rhead[]{\small{1. Introducción}}

\vspace*{0.2cm}
\section{Introducción y objetivos}

Las redes inalámbricas de WiFi se han popularizado en la última década hasta el punto de ser un protocolo barato de usar más a allá de su uso original, con hardware compatible en multitud de dispositivos.

La potencia de su señal, crítica en su uso principal y en otros como el posicionamiento local, es difícil de predecir en escenarios con obstáculos, de forma que aplicaciones que no los tengan en cuenta no podrán garantizar un desempeño correcto.

Para poder conocer esta potencia en las zonas de interés surge la idea de usar la aproximación de las ondas a un rayo que se deplaza con el frente de onda, de modo que es posible conocer su recorrido teniendo en cuenta los posibles rebotes que se puedan producir.

Esta técnica ha sido tradicionalmente usada en el desarrollo y posicionamiento de antenas de telefonía, principalmente para su uso en ciudades.
En estos escenarios, los edificios harán las veces de obstáculos de modo que en ciertas zonas la señal puede ser más débil.

La simulaciones con trazado de rayos buscan encontrar este tipo de zonas antes de su implantación, de modo que sea posible corregir las posibles deficiencias y conseguir un posicionamiento óptimo de las antenas usadas.

En este trabajo se plantea su uso en un escenario local.
Como se comentaba al inicio, la señal de WiFi encontrará en los escenarios donde se usa obstáculos en forma de paredes o grandes objetos que impidan su avance.

Esto generará efectos de propagación multicamino, donde los rebotes de la señal en las paredes y obstáculos que se encuentren en su camino harán que en ciertos puntos lleguen, además de la señal original, una cierta potencia adicional o bien, en escenarios sin visión directa, la señal recibida será exclusivamente la de estos rebotes.

De forma general, la modelización de estos efectos es imposible basándose únicamente en argumentos geométricos o estadísticos, por lo que será necesario simular la propagación de la señal con un mapa virtual del entorno a estudiar. 


%------------------------------------------------------------------------------
%                             Fundamento teórico
%------------------------------------------------------------------------------
\setcounter{table}{0}
\setcounter{equation}{0}
\setcounter{figure}{0}

\newpage
\rhead[]{\small{2. Fundamento teórico y metodología}}

\vspace*{0.2cm}
\section{Fundamento teórico y metodología}

El uso del trazado de rayos como aproximación a ondas electromagnéticas no es una idea reciente, aunque su alto coste computacional ha hecho que su popularidad haya crecido de forma notable en los últimos años al haber sido posible su uso de forma general.

La principal disciplina en busca de implantaciones de esta técnica es la de gráficos generados computacionalmente, en la que se aplican estas aproximaciones a la luz de modo que se puedan conseguir resultados realistas de una forma más sencilla comparado con la rasterización habitual en la industria.

Encontramos en la bibliografía de este ámbito numerosos capítulos dedicados a la implantación del trazado de rayos que han sido recuperados en este trabajo.
Aunque las bases geométricas son similares, las características electromagnéticas de los rayos de luz simplifican parte de los cálculos que debemos corregir en nuestro caso.

\subsection{Geometría}

Comenzamos con la geometría del problema caracterizando los rayos a evaluar.
Su composición es muy simple: constan de un punto de origen y una dirección.
En su transcurso se encontrarán con las paredes del mapa, contra las que rebotarán para seguir su recorrido en la zona de interés.

Para poder modelizar este comportamiento se considera el rayo como una recta.
En este caso el origen $O$ será un punto de esta directa, con su dirección siendo el vector director $\vb{d}$ de la misma de modo que sigue la ecuación
\begin{equation}
    O + t\vb{d}
\end{equation}
con $t\in \mathbb{R}$.

Las paredes serán planos definidos con cuatro puntos como extremos, a partir de los cuáles se calcula su vector normal.
Así, los puntos de intersección de las rectas con alguno de estos planos serán los puntos donde los rayos golpearán las paredes, que servirán de origen para los rayos trasmitidos y reflejados que se generen.

Para determinar cuál es este punto partimos de la consideración de que el vector normal del plano --denominado aquí $\vu{n}$-- será perpendicular a cualquier vector contenido en dicho plano, en este caso el definido como diferencia entre el punto de intersección $X$ y el punto donde está definida la normal\footnote{Esta consideración proviene del caso de una superficie general; en este caso la normal será la misma independientemente de dónde se defina, pudiendo elegir cualquier punto del plano.} $P$ de tal forma que su producto escalar es nulo
\begin{equation}\label{eq:Plano-recta1}
    (X-P)\vdot\vu{n} = 0
\end{equation}

$X$ es un punto de la recta, por lo que debe cumplir, para un cierto $t_i$
\begin{equation}\label{eq:recta_ti}
    X = O + t_i\vb{d}
\end{equation}
que es posible incluir en \eqref{eq:Plano-recta1} de forma que
\begin{equation}
    ((O + t_i\vb{d})-P)\vdot\vu{n} = 0
\end{equation}
que se puede manipular para obtener $t_i$
\begin{equation}\label{eq:t_i}
    \begin{aligned}
        ((O + t_i\vb{d})-P)\vdot\vu{n} &= 0 \\
        (O - P)\vdot\vu{n} + t_i\vb{d}\vdot\vu{n}  &= 0 \\
        t_i\vb{d}\vdot\vu{n} &= - (O - P)\vdot\vu{n}\\
        t_i &= -\frac{(O - P)\vdot\vu{n}}{\vb{d}\vdot\vu{n}}\\
        t_i &= \frac{(P-O)\vdot\vu{n}}{\vb{d}\vdot\vu{n}}
    \end{aligned}
\end{equation}

Con este valor es posible ahora usar la Ec.~\eqref{eq:recta_ti} para obtener las coordenadas del punto de impacto, pero no en cualquier caso.

Es necesario tener varias consideraciones a la hora de determinar $t_i$.
La primera de ellas es que es posible obtener un valor negativo: al modelizar el rayo como una recta se abre la posibilidad de encontrar un punto de intersección en la dirección contraria al vector director, por lo que consideraremos que no hay intersección si $t_i < 0$.

Otra de las posibilidades es que la recta sea paralela al plano, de tal forma que no exista un punto de intersección.
Si esto ocurre, el producto escalar del denominador de la Ec.~\ref{eq:t_i} tomará un valor nulo, por lo que es necesario evitar la operación.

Es más, es posible optimizar este caso en un grado algo mayor al tener en cuenta que ángulos bajos de la normal del vector y la dirección de la recta también indicarán que la intersección se producirá a una distancia muy lejana, por lo que a efectos prácticos no se producirá --habrá otra pared más cerca--.
Así, en el caso de que $\vb{d}\vdot\vu{n} < 10^{-4}$ se interpretará que no hay un punto de intersección con la pared a evaluar.

La última de las consideraciones tiene que ver con los errores de redondeo.
A la hora de calcular $t_i$ o las coordenadas del punto de impacto es posible que el punto de intersección obtenido no se encuentre en el plano de incidencia, lo que provoca futuros errores con los rayos reflejados y transmitidos.
Para evitarlo, solo se considerará la intersección con los planos si $t_i > 10^{-3}$.

Esta condición puede parecer confusa pero su razón se puede ver en la Figura~\ref{fig:condicion_interseccion}.
Como se explicaba, en ocasiones los puntos de intersección no se encuentran exactamente en los planos de incidencia, por lo que alguno de los dos rayos generados encontrarán un nuevo punto de intersección en el mismo plano, pero a una distancia minúscula.

Estos nuevos puntos no solo son erróneos --no se deberían producir--, sino que además pueden llegar a producir valores totalmente distorsionados de la potencia de la señal como se explicará en secciones sucesivas.

\vspace*{2cm}

Una vez definidos los rayos y el entorno solo falta la modelización de los receptores de señal.

La interpretación anterior de rayos y paredes hace que solo se tengan en cuenta los puntos de partida e impacto de las rectas, pero no su camino entre ellos.
Es este camino el objetivo del problema, pues son los puntos en los que la señal de WiFi es útil.

Para solventarlo se colocan a lo largo del entorno generado una serie de esferas que harán las veces de antenas receptoras.
Estas esferas registrarán la potencia de los rayos que impacten contra ellas, de modo que sea posible obtener en los puntos en los que se han colocado la potencia total de señal que una antena colocada en el mismo lugar.

La elección de modelizar estos receptores como esferas parte de los múltiples rayos que impactarán sobre ellas.
Al lanzar un gran número de rayos, todos los receptores recibirán varios --más aún con los sucesivos rebotes--, por lo que no podremos registrarlos todos.

La elección correcta será la del rayo que impacte de forma más directa, ya que el resto de ellos serán residuos fruto del volumen de la esfera que no están presentes en la realidad.

Es posible determinar este rayo teniendo en cuenta en el ángulo de incidencia respecto a la normal de la esfera, pero este paso es evitable teniendo en cuenta que el decaimiento de la potencia es función de la distancia recorrida.

\subsection{Electromagnetismo}

\subsection{Antenas}



%------------------------------------------------------------------------------
%                    Simulación en serie y en paralelo
%------------------------------------------------------------------------------
\setcounter{table}{0}
\setcounter{equation}{0}
\setcounter{figure}{0}

\newpage
\rhead[]{\small{3. Implementación}}

\vspace*{0.2cm}
\section{Simulación en serie y en paralelo}

Partiendo de las bases descritas en la sección anterior la simulación consistirá, de forma general, en lanzar rayos y acumular los resultados de los receptores.

Aprovechando las ventajas de la programación orientada a objetos, se considerarán los siguientes clases:
\begin{itemize}
    \item \textbf{Rayo} Esta clase contendrá el punto de partida y el vector de dirección del rayo, así como la potencia en el origen.
    \item \textbf{Pared} En esta clase se incluirán los cuatro puntos de los extremos de la pared, a partir de los que se calculará su normal, también almacenada. Además, contendrá el valor de permeabilidad dieléctrica de su material.
    \item \textbf{Receptor} Esta clase incluye el origen de la esfera y su radio.
\end{itemize}

Las clases de las paredes y los receptores implementarán métodos que tengan de entrada un cierto rayo y que determinarán si habrá algún impacto entre ellos.

A la hora de almacenar los datos de los receptores no es posible acumular los valores de todos los rayos impactados ya que daría valores muy altos totalmente irreales.
El hecho de que muchos de los rayos impacten contra los receptores es fruto de la modelización como esfera, dando más volumen del que realmente tiene la esfera.

Por ello, se dividirá el reconocimiento de impactos dependiendo de los rebotes que haya recibido.
De esta forma se podrán separar los rayos que se reciben de forma directa de los rebotados.

A la hora de registrar la potencia recibida, para cada uno de los rebotes que ha sufrido el rayo, se elegirá aquel con una potencia mayor, al ser el que impacta de forma más directa con la esfera receptora, comportamiento buscado.

Estos valores se registrarán en una matriz en la que se considera a cada fila como cada receptor, y cada columna como estos valores de rebotes.
Así, tras finalizar la evaluación de rayos se acumularán todos los valores en cada fila, siendo el resultado final para cada receptor.

\begin{algorithm}
    \caption{Bucle Principal}
    \label{euclid}
    \begin{algorithmic}[1]
        \State Leer parámetros.
        \State Cargar mapa.
        \State Colocar receptores.
        \For{azimut $\in [0, 360)$}\Comment{Se evaluan los ángulos sumando un cierto paso fijado en los parámetros.}
            \For{elev $\in [90, -90]$}
                \State eval\_ray(azimut, elev, pot, data)
            \EndFor
        \EndFor

        \For{$i\gets 0, rows(data)$}
            \ForAll{$j\gets 1, cols(data)$}
                \State data[i][0] $\gets$ data[i][0] + data[i][j]
            \EndFor
        \EndFor

        \State write\_results(file, data)
    \end{algorithmic}
\end{algorithm}

\subsubsection*{Evolución de cada rayo}

El lanzamiento y evolución de cada rayo tiene varias fases a discutir.

En primer lugar, una vez se ha determinado la dirección de lanzamiento se debe calcular la potencia a emitir.
Habiendo importado la direccionalidad de la antena de emisión, se realiza una interpolación bidimiensional\footnote{En este caso esta interpolación ha sido bilineal. Aunque es posible obtener unos valores más precisos con métodos de mayor orden, partiendo de puntos separados solo un grado en ambos ángulos esta elección proporciona valores aceptables sin tener que abordar las condiciones de contorno en los extremos.} para determinar el valor exacto.

Como se comentaba en la sección del fundamento teórico, en cada impacto se generarán dos rayos.
Así, si este impacto llega a producirse, será necesario almacenar uno de los rayos producidos para ser evaluado más tarde.

Este almacenamiento se hará en forma de pila, de tal forma que los rayos añadidos más recientemente serán los primeros en ser evaluados.
Esta elección es totalmente arbitraria: todos los rayos son independientes entre sí.

Para optimizar el uso de memoria de la pila se reservará toda la memoria que se podría usar antes de iniciar el bucle y así no necesitar movimientos posteriores.
El tamaño necesario vendrá determinado por el número de rebotes que se quieran registrar.

% \begin{figure}[H]
%     \centering
%     \begin{tikzpicture}

    % Conexiones
    \draw (0,0) -- (-2,-1);
    \draw (0,0) -- (2,-1);

    \draw (-2,-1) -- (-3,-2);
    \draw (-2,-1) -- (-1,-2);

    \draw (-3,-2) -- (-3.5,-3);
    \draw (-3,-2) -- (-2.5,-3);

    \draw[dashed] (2,-1) -- (2,-3);
    \draw[dashed] (-1,-2) -- (-1,-3);


    % Puntos
    \filldraw (0,0) circle [radius=3pt, fill=black];

    \filldraw (-2,-1) circle [radius=3pt, fill=black];
    \filldraw[draw=black, fill=white] (2,-1) circle [radius=3pt];

    \filldraw (-3,-2) circle [radius=3pt, fill=black];
    \filldraw[draw=black, fill=white] (-1,-2) circle [radius=3pt];

    \filldraw (-3.5,-3) circle [radius=3pt, fill=black];
    \filldraw[draw=black, fill=white] (-2.5,-3) circle [radius=3pt];

    % Niveles
    \draw[->] (4,0) -- (4,-1);
    \node[anchor=west] at (4,0) {0};

    \draw[->] (4,-1) -- (4,-2);
    \node[anchor=west] at (4,-1) {1};


    \draw[dashed, ->] (4,-2) -- (4,-3);
    \node[anchor=west] at (4,-3) {$h$};

    % Envoltorio

    \draw[dashed] [rounded corners=0.5cm] (1.75,0.5) -- (-3.5,-3.25)[rounded corners=0.5cm] -- (3.5,-3.25) -- cycle;


\end{tikzpicture}
% \end{figure}

En la Figura~\ref{fig:memoria_pila} se indica el árbol generado por un rayo, donde en cada impacto nacen dos rayos.
Considerando que siempre se evalúa el rayo reflejado --en esta representación, el camino de la izquierda-- podría parecer que se deberá reservar memoria para todos los rayos restantes, pero es posible un uso menor.

Los rayos guardados en la pila no son evaluados de forma inmediata, así que los rayos que se generan no se conocerán hasta que se llegue a su nodo.
Es decir, el número de rayos en la pila no llegará nunca a ser tan alto.


\begin{algorithm}
    \caption{Bucle que evalúa cada rayo}
    \label{euclid}
    \begin{algorithmic}[1]
        % \State Añadir pareja \{rayo inicial, 0\} a la pila.
        
        \While{Tamaño pila > 0}
            \State reb $\gets$ segundo valor de la pareja \{rayo, rebote\}.

            \ForAll{Paredes}
                \State hit\_dist $\gets$ INFINITY
                \State wall\_hit $\gets$ -1
                \If{Pared golpeada}
                    \If{dist < hit\_dist}
                        \State hit\_dist $\gets$ dist
                        \State wall\_hit $\gets$ pared
                    \EndIf
                \EndIf
            \EndFor

            \ForAll{Receptores}
                \State hit\_dist $\gets$ INFINITY
                \State power $\gets$ 0
                \If{Receptor golpeado}
                    \If{hit\_power > power and dist < hit\_dist}
                        \State hit\_power $\gets$ power
                    \EndIf
                \EndIf
            \EndFor

            \If{power > CUTOFF\_POWER AND reb < MAX\_REBOUND}
                \State \{rayo, rebote\} $\gets$ \{ rayo\_reflejado, reb+1\}
                \State Añadir \{ rayo\_transmitido, reb+1\} a la pila.
            \Else
                \State \{rayo, rebote\} $\gets$ último valor de la pila.
            \EndIf
        \EndWhile
    \end{algorithmic}
\end{algorithm}

\subsection{Implementación en paralelo}

El hecho de que los rayos lanzados sean independientes entre sí convierte en el bucle principal en el escenario perfecto para ser ejecutado de forma paralela, ya que cada iteración del bucle no se verá interrumpida por las demás.

Para esta paralelización se ha hecho uso de tarjetas gráficas, que cuentan con un gran número de núcleos de cómputo pero para las que hay que tener en cuenta su estructura de memoria.

En general, a la hora de paralelizar un algoritmo es necesario hacer duplicaciones de datos para evitar condiciones de carrera que proporcionen resultados erróneos.
En los casos de muy alta paralelización como este es posible encontrarse con ciertas limitaciones.

La tarjeta gráfica usada en este caso ha sido una Nvidia GTX1070, sobre la que es posible el uso de la librería de Nvidia CUDA.
Esta tarjeta cuenta con 1920 núcleos agrupados en ??? multiprocesadores.

La estrategia de paralelización consistirá en asignar un rayo a cada hilo, que contará con su matriz de datos correspondiente donde registrará los impactos para luego ser acumulada en los datos finales.

Las tarjetas gráficas agrupan sus hilos en bloques.
Los hilos que se encuentren en un mismo bloque pueden acceder a una cierta cantidad de memoria compartida de un acceso más rápido que la memoria global común.

Para reducir la necesidad de memoria y aprovechar esta característica la acumulación de los datos se producirá en dos fases: el cálculo de cada rayo --asignado a cada hilo-- volcará sus datos en la memoria compartida del bloque.
Una vez acaban todo los hilos del bloque, se acumulan los datos en la memoria global, donde se ha reservado espacio para cada bloque.

Debido a la herencia del uso de este tipo de dispositivos para trabajos de vídeo, es posible acceder a los hilos y bloques con dos índices, de forma similar a una matriz.
Esta característica será útil en este caso, en el que los rayos se lanzarán en torno a los dos ángulos de las coordenadas esféricas.

Así, se asociará la dirección $x$ al ángulo azimutal y la dirección $y$ a la elevación.

Queda determinar el número de hilos en cada bloque.
En general, la memoria disponible para su uso compartido es una cantidad baja.
En el caso de la tarjeta gráfica usada, es de 48 KB\cite{Nvidia}.

\begin{figure}[H]
    \centering
    \begin{tikzpicture}
    \draw[line width=2pt] (0,0) -- (14,0);
    \draw[line width=2pt] (0,-6.5) -- (14,-6.5);
    \draw[line width=2pt] (0,0) -- (0,-6.5);
    \draw[line width=2pt] (14,0) -- (14,-6.5);

    % Labels
    \foreach \i in {0, 1.5, 3,...,10}
    {
        \draw (\i,0.3) -- (\i,-6.5);
        \pgfmathtruncatemacro{\label}{4*\i/1.5};
        \node[anchor=south] at (\i, 0.3) {\label \si{\degree}};
    }
    \draw (12.5,0.3) -- (12.5,-6.5);
    \node[anchor=south] at (12.5, 0.3) {356\si{\degree}};
    \node[anchor=south] at (10.75, 0.3) {\ldots};

    \foreach \i in {0, -1.5, -3}
    {
        \draw (-0.3, \i) -- (14, \i);
        \pgfmathtruncatemacro{\label}{90+4*\i/1.5};
        \node[anchor=east] at (-0.3, \i) {-\label \si{\degree}};
    }

    \draw (-0.3, -5) -- (14, -5);
    \node[anchor=east] at (-0.5, -4) {\vdots};
    \node[anchor=east] at (-0.3, -5) {86\si{\degree}};

    %Blocks
    \foreach \i in {1.5, 3, ...,10}
    {
        \pgfmathtruncatemacro{\bloquei}{\i/1.5 - 1};
        \foreach \j in {-1.5, -3}{
            \pgfmathtruncatemacro{\bloquej}{-\j/1.5-1};
            \node[anchor=south] at (\i-0.75, \j+0.75) {\small{Bloque}};
            \node[anchor=north] at (\i-0.75, \j+0.75) {\small{(\bloquei, \bloquej)}};

        }
        % \draw (\i,0.3) -- (\i,-6.5);
        % \pgfmathtruncatemacro{\label}{4*\i/1.5};
        
    }

    % Bloques extremo izquerda
    \node[anchor=south] at (13.25, -0.75) {\small{Bloque}};
    \node[anchor=north] at (13.25, -0.75) {\small{(89, 0)}};

    \node[anchor=south] at (13.25, -2.25) {\small{Bloque}};
    \node[anchor=north] at (13.25, -2.25) {\small{(89, 1)}};

    \node[anchor=south] at (13.25, -5.75) {\small{Bloque}};
    \node[anchor=north] at (13.25, -5.75) {\small{(89, 44)}};


    %Puntitos
    \foreach \i in {1.5, 3, ..., 10, 14}
    {
        \node[anchor=east] at (\i-0.5, -4) {\vdots};
    }

    \foreach \j in {0, -1.5, -5}
    {
        \node[anchor=east] at (11.25, \j-0.75) {\ldots};
    }

    \node[anchor=east] at (11.25, -4) {$\ddots$};


\end{tikzpicture}
    \caption{aa}
    \label{fig:CUDA_angulos}
\end{figure}

Esta cantidad no permite lanzar bloques con un número grande de hilos, aunque dependerá del número de receptores y los rebotes que se quieran registrar.
A modo de ejemplo, al colocar unos 40 receptores y registrar los rayos hasta 5 rebotes, con valores de precisión simple --es decir, 4 bytes--, podremos lanzar bloques con 60 hilos como máximo.

Para evitar futuras incompatibilidades en caso de elegir valores mayores en alguno de estos dos parámetros, se han elegido el mínimo valor de hilos posible.
La elección es la siguiente: solo se lanzarán 16 hilos por bloque, 4 en cada dirección.

Además de ser un valor bajo compatible con los límites establecidos, el número de ángulos a evaluar en ambas direcciones es divisible por 4, hecho aprovechable para la implementación.

La elección de asignación de valores angulares a los bloques será de forma fija: cada bloque solo evaluará 4 grados en cada dirección, independientemente del número de rayos que tenga asignado ejecutar.

Está claro que si la distancia entre los ángulos de un grado, cada hilo ejecutará el rayo con los ángulos que le corresponde sin más.
En el caso --habitual-- que la distancia sea menor, cada hilo deberá evaluar más de un rayo.

Para facilitar el recorrido y garantizar una ejecución lo más homogénea posible\footnote{Hay que tener en cuenta que la ejecución de los programas en tarjetas gráficas se realizan ejecutando una misma instrucción en varios hilos a la vez. Evitar expresiones condicionales que creen distintas ramas en el desarrollo del programa maximizará el rendimiento del dispositivo.}, solo se considerarán saltos en los valores angulares como potencias negativas de dos --es decir, solo se usarán saltos de valor $0.5$, $0.25$, $0.125$,...-- a fin de poder utilizar la siguiente estrategia.

Se considerará una matriz de «minibloques» dentro del bloque, todas ellas con los 16 hilos del bloque.
La cantidad de estos minibloques dependerá del valor del salto entre ángulos: si es $0.5 \equiv \sfrac{1}{2}$ habrá 4 minibloques, dos en cada dirección; si es $0.25 \equiv \sfrac{1}{4}$ habrá 16 minibloques, etc.

Dentro de cada uno de los minibloques solo habrá 16 pares de direcciones a evaluar, uno para cada hilo.
Una vez todos estos hilos finalicen su bucle, se avanzará al siguiente minibloque en dirección vertical --es decir, avanzando en el eje $y$-- hasta acabar con todos los ángulos que se deben evaluar.

%------------------------------------------------------------------------------
%                             Entorno de pruebas
%------------------------------------------------------------------------------
\setcounter{table}{0}
\setcounter{equation}{0}
\setcounter{figure}{0}

\newpage
\rhead[]{\small{4. Entorno de pruebas}}

\newpage
\vspace*{0.2cm}
\section{Entorno de pruebas}

Para comprobar cómo se ajusta el modelo al comportamiento real de la emisión se tomaron medidas en el Laboratorio de Robótica 0L3 del Instituto de Computación Científica Avanzada de la Universidad de Extremadura (ICCAEx), situado en los Institutos Universitarios de Investigación de la Universidad de Extremadura en Badajoz.

En este laboratorio se dispone de una superficie amplia en la que fue posible usar un robot para automatizar la toma de medidas en divesos puntos, que más tarde fueron usados en la simulación colocando los receptores en dichos puntos.

\begin{figure}[H]
    \centering
    \begin{subfigure}[b]{0.45\textwidth}
        \centering
        \includegraphics[width=6.8cm]{pic/lab1.jpg}
        % \caption{Imágenes de la antena del router utilizado.}
        % \label{fig:antena_router}
    \end{subfigure}
    ~~
    \begin{subfigure}[b]{0.45\textwidth}
        \centering
        \includegraphics[width=6.8cm]{pic/lab2.jpg}
        % \caption{Imágenes de la antena receptora utilizada}
        % \label{fig:antena_receptor}
    \end{subfigure}
    \caption{Imágenes del laboratorio donde se realizaron las pruebas.}
    \label{fig:lab_pics}
\end{figure}


Cabe destacar que el entorno de pruebas tenía un especto electromagnético muy poblado.
Al encontrarse en un edificio con otros laboratorios, se encontraban entre 10 y 15 redes WiFi, con lo que resultó imposible encontrar algún canal libre para realizar las pruebas.

Para solventar esta contaminación se tomaron, para cada punto, 5 medidas esperando 5 segundos ente cada una de ellas, de modo que fue posible, tomando medias de las mismas, retirar los posibles efectos de inteferencia provocados por el resto de redes.

% Para una mayor comprobación en el rendimiento del simulador, se utilizaron unas tablas de madera de dimensiones $1 \times 0.6 \times 0.04\si{\meter}$ forradas de forma parcial con papel de alumninio, de forma que pudieran presentar un cierto efecto de apantallamiento de la radiación en su dirección.
% En subsecciones posteriores se explicará en detalle su disposición.

% Así, en cada punto de la trayectoria el robot tomó 5 medidas, separadas 1 segundo entre ellas.
% Además, se repitió la trayectoria no menos de 10 veces para cada una de las configuraciones de prueba.

\subsection{Robot}

Con el fin de automatizar la toma de medidas para hacerla lo más eficiente posible se ha decidido usar un robot móvil capaz de desplazarse mediante navegación autónoma en un entorno conocido.

El elegido en este caso fue el robot TurtleBot 2, un robot con fines educativos y de investigación capaz de desplazarse y orientarse con total libertad en superficies llanas, como eran los en el que se desarrolló este trabajo.
En la Figura~\ref{fig:robot} se puede ver el robot durante una de las tomas de medidas.

\begin{figure}[H]
    \centering
    \includegraphics[width=0.55\textwidth]{pic/robot_lab.jpg}
    \caption{Turtlebot en una de las tomas de medidas.}
    \label{fig:robot}
\end{figure}

El TurtleBot funciona con ROS --\textit{Robot Operating System}, en inglés--, un entorno de trabajo enfocado a los sistemas robóticos encargado del soporte para todos los dispositivos hardware del robot como motores, encoders o cámaras. 
Cuenta con diversas librerías para distintos casos de uso, entre las que se encuentra la navegación en un entorno controlado como la que se da en el caso de este trabajo.

Su funcionamiento se basa en una arquitectura de grafos, donde se definen \textit{nodos}, tareas en las que se realiza el procesamiento de sensores, control, actuadores o cualquier otra función.
Los nodos se comunican entre ellos con mensajes llamados \textit{topic}, y es posible su desarrollo con los lenguajes C++ y Python.

Dentro de los paquetes disponibles, el usado en el desarrollo del trabajo es el encargado de la navegación del robot, que permite desplazar al robot a cualquier punto del entorno de trabajo y proporcionar de forma constante su posición en el mapa.

Es posible acoplar al robot un sistema de visión llamado Kinect que, mediante luz, permite obtener un mapa de profundidad del entorno que lo rodea.
Fue desarrollado en primera instancia para el uso en videojuegos pero su uso también se ha extendido a labores de investigación al permitir el posicionamiento de objetos y paredes, de tal manera que permite al robot evitar obstáculos dinámicos en el caso de que interpongan en su camino.

La funcionalidad de navegación aprovecha estos sensores realizando una fusión de sus resultados con los datos de movimiento de los motores que impulsan al robot, de tal forma que es posible corregir cualquier error en casos donde el empuje de las ruedas no se traslade directamente en un desplazamiento del robot, como puede ocurrir al rotar sobre sí mismo.

Para conseguir el posicionamiento en el mapa el paquete de navegación utiliza un \textit{planner} global sobre el que es posible determinar las rutas a seguir por el robot para llegar a un punto dado sorteando obstáculos y paredes.
Junto a él trabaja un \textit{planner} local, en el que gracias a los sensores incorporados se evalúan de forma continua los alrededores del robot.
Así, es posible conseguir de forma continua una evaluación de los posibles obstáculos que no se encuentran en el mapa del planner global, para lo que se construye un mapa de costes con el fin de abortar el movimiento en el caso de que sea imposible alcanzar el punto objetivo \cite{ROSDoc}.

Con la información actualizada del \textit{planner} local, el \textit{planner} global es capaz de modificar los trayectos para que el robot pueda continuar su desplazamiento por el mapa.
Es posible observar un esquema del funcionamiento de estos sistemas en la Figura~\ref{fig:move_base}.

\begin{figure}[H]
    \centering
    \def\svgwidth{0.8\linewidth}
    \input{./fig/navigation.pdf_tex}
	\caption{Esquema del funcionamiento del paquete de navegación de ROS.}
    \label{fig:move_base}
\end{figure}

Con esta herramienta la librería de navegación permite un mapeado autónomo tomando como referencia los datos de odometría que proporcionan los motores propulsores del robot para determinar las dimensiones del entorno.

En el caso de disponer de antemano de dichas dimensiones es posible proporcionar un mapa al sistema de navegación y evitar el paso de reconocimiento del entorno.
Esto no solo ahorra tiempo, sino que además minimiza las posibles discrepancias entre los datos de odometría y los desplazamientos reales del robot al realizar el mapeado de forma autónoma.

La opción de realizar un mapa previo fue la elegida en este caso, ya que el robot cuenta con rutinas para el reposicionamiento en el mapa en dicho caso.
Así, a partir de los límites establecidos y comprobados de forma manual, las posibles discrepancias en la odometría del robot se ven continuamente compensadas y corregidas.

A partir de estos datos corregidos es posible conocer en cualquier momento la posición del robot en el mapa, expuesta a través de ROS en uno de los topic disponibles.

Para facilitar el uso de ROS existe la posibilidad de usar el simulador Stage, capaz de crear un mundo virtual a partir de un mapa en dos dimensiones en el que colocar el Turtlebot y simular su funcionamiento de forma total sin tener acceso al robot de forma física.
Es posible observar su interfaz en la Figura~\ref{fig:stage_rviz}.

Además, ROS también permite el uso de una herramienta de visualización de la posición del robot en el mapa y de todos los sensores que incorpora llamada \textit{rviz}.
Aunque es compatible con Stage, sus funcionalidades brillan al usar el robot en entornos reales, donde es posible comprobar de forma continua que su posicionamiento es correcto y que sus sensores funcionan como es debido.


\begin{figure}[H] 
    \centering
    \includegraphics[width=0.75\textwidth]{pic/Stage-rviz.png}
    \caption{Captura del simulador Stage (a la izquierda) y la herramienta de visualización \textit{rviz} (a la derecha).}
    \label{fig:stage_rviz}
\end{figure}

\subsubsection{Trayectoria}

Dentro del laboratorio se eligió un área de $7 \times 5 \si{\meter}$ para el recorrido del robot, eligiendo los puntos en los que tomar las medidas separados 1 metro.

\begin{figure}[H]
    \begin{subfigure}[b]{.45\textwidth}
        \centering
        \def\svgwidth{0.6\linewidth}
	    \input{./fig/lab.pdf_tex} 
        \caption{Puntos a evaluar.}
        \label{fig:puntos}
    \end{subfigure}
    \begin{subfigure}[b]{.45\textwidth}
        \centering
        \def\svgwidth{0.6\linewidth}
	    \input{./fig/lab_vertical.pdf_tex} 
        \caption{Trayectoria del robot.}
        \label{fig:vertical}
    \end{subfigure}
    \caption{Puntos de medida por el robot en el laboratorio y la trayectoria tomada.}
    \label{fig:laboratorio}
\end{figure}

La Figura~\ref{fig:laboratorio} recoge la disposición de los puntos tomados y la ruta del robot para ello.
En total se compone de 35 puntos: en cada uno de ellos el robot se parará y tomará las 5 medidas correspodientes.

En todos los escenarios a comentar en la sección posterior la trayectoria fue la misma.
Si se coloca algún obstáculo en el camino del robot lo podrá evitar de forma totalmente autónoma, evitando tener que hacer cualquier modificación en la trayectoria establecida y programada.

\subsection{Escenarios}

Además del escenario del laboratorio sin ningún obstáculo, se utilizaron unas tablas de madera de dimensiones $1 \times 0.64 \times 0.02\si{\meter}$ forradas de forma parcial con papel de alumninio, de forma que pudieran presentar un cierto efecto de apantallamiento de la radiación en su dirección.

Estas tablas, colocadas en dos configuraciones, proporcionaron otros dos escenarios con obstáculos para comprobar de manera más profunda el comportamiento del simulador.

En todos los casos el emisor estuvo colocado en la misma posición: un metro por detrás del punto 2 tal y como se indica en la Figura~\ref{fig:laboratorio}, a $0.55\si{\centi\meter}$ de altura.

Es esta altura por la que solo se forra la mitad superior de las tablas: tal y como se mostraba en la sección~\ref{sec:antenas}, el patrón de emisión coloca la mayor parte de la potencia siendo emitida en el plano de la antena.
Sumando el mayor decaimiento al recorrer una distancia mayor, las trayectorias alternativas que involucren rebotes contra el suelo o el techo tendrán un efecto mucho menor, así que se conseguirá un apantallamiento suficiente con esta configuración.

\subsubsection{Escenario A}

El primer escenario fue la utilización del laboratorio sin ningún obstáculo, de forma que se espera que los rayos impacten únicamente contra las paredes, puertas y demás mobiliario que se queda fuera de la trayectoria del robot.

Por ello, en este caso predominará el efecto del decaimiento con la distancia, salvo en puntos cercanos a los límites del laboratorio.

\subsubsection{Escenario B}

Para un segundo escenario se colocó una tabla forrada tal y como se describía anteriormente.

Se colocó centrada horizontalmente, 1 metro por debajo del centro de la trayectoria como se muestra en la Figura~\ref{fig:escenarioB}.

\begin{figure}[H]
    \centering
    \begin{subfigure}[b]{0.45\textwidth}
        \centering
        % \includegraphics[width=6.8cm]{pic/lab1.jpg}
        \begin{tikzpicture}[scale=0.90]
    \draw (-2,-3) rectangle (2,3);

    \draw[thick] (-0.5, -1) -- (0.5, -1);

    \fill[black] (0,0) circle (0.05);

    \fill[red] (-1,-3.5) circle (0.075);

    \draw[<->] (0,-0.1) -- (0,-0.9);
    \node[anchor=west] at (0,-0.5) {1m};

    \draw[<->] (-1.75,-2.9) -- (-1.75,2.9);
    \node[anchor=west] at (-1.75,-2.5) {7m};

    \draw[<->] (-1.75, 2.7) -- (1.75, 2.7);
    \node[anchor=north] at (0, 2.7) {5m};
\end{tikzpicture}
        \caption{Esquema de la posición de la tabla. El punto rojo indica la posición del emisor.}
        % \label{fig:antena_router}
    \end{subfigure}
    ~~
    \begin{subfigure}[b]{0.45\textwidth}
        \centering
        \includegraphics[width=6.8cm]{pic/escB.jpg}
        \caption{Imagen del laboratorio en el escenario B.}
        % \label{fig:antena_receptor}
    \end{subfigure}
    \caption{Diposición en el escenario B.}
    \label{fig:escenarioB}
\end{figure}

En este caso se busca que la tabla presente un efecto atenuante en toda la zona detrás de la tabla, creando una zona diagonal donde se deberían obtener valores de potencia menores que en el escenario A.

\subsubsection{Escenario C}

El tercer y último escenario tiene en cuenta dos tablas, esta vez colocadas con una orientación distinta.

Se colocan en este caso dos tablas a $1.5$ metros del centro de distancia horizontal, a ambos lados.
En cuanto a la distancia vertical, una de ellas se coloca a $1.5$ metros por encima y la otra la misma distancia por debajo. --donde esta orientación hace referencia a la distribución indicada en la Figura~\ref{fig:escenarioC}--.

\begin{figure}[H]
    \centering
    \begin{subfigure}[b]{0.45\textwidth}
        \centering
        % \includegraphics[width=6.8cm]{pic/lab1.jpg}
        \begin{tikzpicture}[scale=0.90]
    \draw (-2,-3) rectangle (2,3);

    \draw[thick] (-1.5, 1) -- (-1.5, 2);
    \draw[thick] (1.5, -1) -- (1.5, -2);

    \fill[black] (0,0) circle (0.05);

    \fill[red] (-1,-3.5) circle (0.075);

    \draw[<->] (0,-0.1) -- (0,-1.4);
    \draw[dashed] (0, -1.5) -- (1.4, -1.5);
    \node[anchor=east] at (0,-0.75) {1.5m};

    \draw[<->] (0, 0.1) -- (0, 1.4);
    \draw[dashed] (0, 1.5) -- (-1.4, 1.5);
    \node[anchor=west] at (0, 0.75) {1.5m};

    \draw[<->] (-1.75,-2.9) -- (-1.75,2.9);
    \node[anchor=west] at (-1.75,-2.5) {7m};

    \draw[<->] (-1.75, 2.7) -- (1.75, 2.7);
    \node[anchor=north] at (0, 2.7) {5m};
\end{tikzpicture}
        \caption{Esquema de la posición de las tablas. El punto rojo indica la posición del emisor.}
        % \label{fig:antena_router}
    \end{subfigure}
    ~~
    \begin{subfigure}[b]{0.45\textwidth}
        \centering
        \includegraphics[width=6.8cm]{pic/escC.jpg}
        \caption{Imagen del laboratorio en el escenario C.}
        % \label{fig:antena_receptor}
    \end{subfigure}
    \caption{Diposición en el escenario C.}
    \label{fig:escenarioC}
\end{figure}

A diferencia del caso anterior, en este escenario no se busca tanto el apantallamiento --aunque puede haberlo-- sino el efecto de los rebotes.

Así, se espera que en la parte central pueda haber una potencia algo mayor que en el escenario A, aunque no debería haber un aumento significativo debido a la mayor distancia recorrida por los ángulos y el rebote que sufren.



%------------------------------------------------------------------------------
%                             Resultados
%------------------------------------------------------------------------------
\setcounter{table}{0}
\setcounter{equation}{0}
\setcounter{figure}{0}

\newpage
\rhead[]{\small{5. Resultados}}

\newpage
\vspace*{0.2cm}
\section{Resultados}

\subsection{Entorno de pruebas}

Para comprobar cómo se ajusta el modelo al comportamiento real de la emisión se tomaron medidas en el Laboratorio de Robótica 0L3 del Instituto de Computación Científica Avanzada de la Universidad de Extremadura (ICCAEx), situado en los Institutos Universitarios de Investigación de la Universidad de Extremadura en Badajoz.

En este laboratorio se dispone de una superficie amplia en la que fue posible usar un robot para automatizar la toma de medidas en divesos puntos, que más tarde fueron usados en la simulación colocando los receptores en dichos puntos.

\begin{figure}[H]
    \begin{subfigure}[b]{.45\textwidth}
      \centering
      \def\svgwidth{0.6\linewidth}
	    \input{./fig/lab.pdf_tex} 
      \caption{Puntos a evaluar}
      \label{fig:puntos}
    \end{subfigure}
    \begin{subfigure}[b]{.45\textwidth}
        \centering
        \def\svgwidth{0.6\linewidth}
	    \input{./fig/lab_vertical.pdf_tex} 
        \caption{Trayectoria en vertical}
        \label{fig:vertical}
      \end{subfigure}
    \caption{Puntos de medida por el robot en el laboratorio y la trayectoria tomada.}
    \label{fig:laboratorio}
\end{figure}

La Figura~\ref{fig:laboratorio} recoge la disposición de los puntos tomados y la ruta del robot para ello.
En total se compone de 35 puntos en los el robot se para y toma medidas cada 5 segundos.

Esta trayectoria se repitió unas 20 veces de modo que fue posible hacer estimaciones sobre todos los puntos con un mayor volumen de datos que pudiera eliminar los efectos de otras redes del edificio.

\subsection{Robot}

Con el fin de automatizar la toma de medidas para hacerla lo más eficiente posible se ha decidido usar un robot móvil capaz de desplazarse mediante navegación autónoma en un entorno conocido.

El elegido en este caso fue el robot TurtleBot 2, un robot con fines educativos y de investigación capaz de desplazarse y orientarse con total libertad en superficies llanas, como eran los en el que se desarrolló este trabajo.
En la Figura~\ref{fig:robot} se puede ver el robot durante una de las tomas de medidas.

\begin{figure}[H]
    \centering
    \includegraphics[width=0.55\textwidth]{pic/robot_lab.jpg}
    \caption{Turtlebot en una de las tomas de medidas.}
    \label{fig:robot}
\end{figure}

El TurtleBot funciona con ROS --\textit{Robot Operating System}, en inglés--, un entorno de trabajo enfocado a los sistemas robóticos encargado del soporte para todos los dispositivos hardware del robot como motores, encoders o cámaras. 
Cuenta con diversas librerías para distintos casos de uso, entre las que se encuentra la navegación en un entorno controlado como la que se da en el caso de este trabajo.

Su funcionamiento se basa en una arquitectura de grafos, donde se definen \textit{nodos}, tareas en las que se realiza el procesamiento de sensores, control, actuadores o cualquier otra función.
Los nodos se comunican entre ellos con mensajes llamados \textit{topic}, y es posible su desarrollo con los lenguajes C++ y Python.

Dentro de los paquetes disponibles, el usado en el desarrollo del trabajo es el encargado de la navegación del robot, que permite desplazar al robot a cualquier punto del entorno de trabajo y proporcionar de forma constante su posición en el mapa.

Es posible acoplar al robot un sistema de visión llamado Kinect que, mediante luz, permite obtener un mapa de profundidad del entorno que lo rodea.
Fue desarrollado en primera instancia para el uso en videojuegos pero su uso también se ha extendido a labores de investigación al permitir el posicionamiento de objetos y paredes, de tal manera que permite al robot evitar obstáculos dinámicos en el caso de que interpongan en su camino.

La funcionalidad de navegación aprovecha estos sensores realizando una fusión de sus resultados con los datos de movimiento de los motores que impulsan al robot, de tal forma que es posible corregir cualquier error en casos donde el empuje de las ruedas no se traslade directamente en un desplazamiento del robot, como puede ocurrir al rotar sobre sí mismo.

Para conseguir el posicionamiento en el mapa el paquete de navegación utiliza un \textit{planner} global sobre el que es posible determinar las rutas a seguir por el robot para llegar a un punto dado sorteando obstáculos y paredes.
Junto a él trabaja un \textit{planner} local, en el que gracias a los sensores incorporados se evalúan de forma continua los alrededores del robot.
Así, es posible conseguir de forma continua una evaluación de los posibles obstáculos que no se encuentran en el mapa del planner global, para lo que se construye un mapa de costes con el fin de abortar el movimiento en el caso de que sea imposible alcanzar el punto objetivo \cite{ROSDoc}.

Con la información actualizada del \textit{planner} local, el \textit{planner} global es capaz de modificar los trayectos para que el robot pueda continuar su desplazamiento por el mapa.
Es posible observar un esquema del funcionamiento de estos sistemas en la Figura~\ref{fig:move_base}.

\begin{figure}[H]
    \centering
    \def\svgwidth{0.8\linewidth}
    \input{./fig/navigation.pdf_tex}
	\caption{Esquema del funcionamiento del paquete de navegación de ROS.}
    \label{fig:move_base}
\end{figure}

Con esta herramienta la librería de navegación permite un mapeado autónomo tomando como referencia los datos de odometría que proporcionan los motores propulsores del robot para determinar las dimensiones del entorno.

En el caso de disponer de antemano de dichas dimensiones es posible proporcionar un mapa al sistema de navegación y evitar el paso de reconocimiento del entorno.
Esto no solo ahorra tiempo, sino que además minimiza las posibles discrepancias entre los datos de odometría y los desplazamientos reales del robot al realizar el mapeado de forma autónoma.

La opción de realizar un mapa previo fue la elegida en este caso, ya que el robot cuenta con rutinas para el reposicionamiento en el mapa en dicho caso.
Así, a partir de los límites establecidos y comprobados de forma manual, las posibles discrepancias en la odometría del robot se ven continuamente compensadas y corregidas.

A partir de estos datos corregidos es posible conocer en cualquier momento la posición del robot en el mapa, expuesta a través de ROS en uno de los topic disponibles.

Para facilitar el uso de ROS existe la posibilidad de usar el simulador Stage, capaz de crear un mundo virtual a partir de un mapa en dos dimensiones en el que colocar el Turtlebot y simular su funcionamiento de forma total sin tener acceso al robot de forma física.
Es posible observar su interfaz en la Figura~\ref{fig:stage_rviz}.

Además, ROS también permite el uso de una herramienta de visualización de la posición del robot en el mapa y de todos los sensores que incorpora llamada \textit{rviz}.
Aunque es compatible con Stage, sus funcionalidades brillan al usar el robot en entornos reales, donde es posible comprobar de forma continua que su posicionamiento es correcto y que sus sensores funcionan como es debido.

\begin{figure}[H] 
    \centering
    \includegraphics[width=0.75\textwidth]{pic/Stage-rviz.png}
    \caption{Captura del simulador Stage (a la izquierda) y la herramienta de visualización \textit{rviz} (a la derecha).}
    \label{fig:stage_rviz}
\end{figure}


%------------------------------------------------------------------------------
%                             Conclusiones
%------------------------------------------------------------------------------

\newpage
\vspace*{0.2cm}
\section{Conclusiones}

En este trabajo se ha desarrollado un modelo de emisión electromagnética con la aproximación de trazado de rayos para la modelización de propagación en interiores de señales de redes WiFi.

El uso de esta técnica se basa en el uso de rayos que se propagan en línea recta simulando el comportamiento del frente de ondas, de modo que se producirán rebotes en paredes y obstáculos.
Esta característica la convierte en idónea para el caso de simulación en interiores como este caso.

Debido a la naturaleza independiente de los rayos se planteó la implementación en paralelo de modo que se puedan conseguir aceleraciones en los tiempos de ejecución de la simulación.
Para ello se usaron GPUs, dispositivos idóneos para este tipo de tareas tan altamente paralelizables.
Con ellas se pudieron conseguir tiempos de ejecución hasta 7 veces más bajos respecto a la ejecución en serie en un procesador usual.

Para determinar el comportamiento del simulador se compararon los resultados obtenidos con medidas en un entorno real tomadas con la ayuda de un robot.

Aunque en general los resultados se ajustan de forma razonable, la caracterización de los materiales del entorno resultó ser crítica a la hora de hacer que los rayos se comporten de la misma forma que lo hace la emisión en la realidad.

El entorno de pruebas estaba rodeado de otras redes, lo que contaminaba las medidas tomadas.
Por ello es posible que la discrepancia entre la simulación y las medidas sea aparentemente mayor de lo que debería ser.

Con una buena parametrización de los materiales el decaimiento de la potencia con la distancia se ajusta bastante bien en la realidad, por lo que se podrían usar este tipo de simulaciones sin problemas en los entornos locales a los que están dirigidos.

La posibilidad de obtener el comportamiento de las redes WiFi en entornos locales, ampliamente usadas, abre la puerta a multitud de aplicaciones, desde la planificación de coberturas en entornos con muchos obstáculos hasta su uso para labores de posicionamiento local.

El hecho de poder automatizar la obtención de estos datos facilita estas labores enormemente, eliminando la necesidad de tomar medidas en la zona de interés.
Por ello, este método es más económico y rápido, además de permitir la obtención de datos desde prácticamente cualquier punto del entorno a evaluar.

Un modelo de interacción de los rayos más complejo que tenga en cuenta otros fenómenos como el la difracción, el \textit{scattering} o incluso posibles absorciones de energía por parte de materiales podrían arrojar resultados más ajustados a la realidad, pero implicaría una mayor complejidad y un mayor coste computacional.


%------------------------------------------------------------------------------
%                             Bibliografía
%------------------------------------------------------------------------------

\newpage
\vspace*{0.2cm}

\section{Bibliografía}

\begingroup
\renewcommand{\section}[2]{}%
\bibliography{refs} 
\bibliographystyle{ieeetr}
\endgroup

%------------------------------------------------------------------------------
%                             Anexo
%------------------------------------------------------------------------------

\newpage
\vspace*{0.2cm}
\section*{Anexo}

\input{tex/anexo.tex}
\end{document}