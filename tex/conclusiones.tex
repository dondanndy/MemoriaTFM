En este trabajo se ha desarrollado un modelo de emisión electromagnética con la aproximación de trazado de rayos para la modelización de propagación en interiores de señales de redes WiFi.

El uso de esta técnica se basa en el uso de rayos que se propagan en línea recta simulando el comportamiento del frente de ondas, de modo que se producirán rebotes en paredes y obstáculos.
Esta característica la convierte en idónea para el caso de simulación en interiores como este caso.

Debido a la naturaleza independiente de los rayos se planteó la implementación en paralelo de modo que se puedan conseguir aceleraciones en los tiempos de ejecución de la simulación.
Para ello se usaron GPUs, dispositivos idóneos para este tipo de tareas tan altamente paralelizables.
Con ellas se pudieron conseguir tiempos de ejecución hasta 7 veces más bajos respecto a la ejecución en serie en un procesador usual.

Para determinar el comportamiento del simulador se compararon los resultados obtenidos con medidas en un entorno real tomadas con la ayuda de un robot.

Aunque en general los resultados se ajustan de forma razonable, la caracterización de los materiales del entorno resultó ser crítica a la hora de hacer que los rayos se comporten de la misma forma que lo hace la emisión en la realidad.

El entorno de pruebas estaba rodeado de otras redes, lo que contaminaba las medidas tomadas.
Por ello es posible que la discrepancia entre la simulación y las medidas sea aparentemente mayor de lo que debería ser.

Con una buena parametrización de los materiales el decaimiento de la potencia con la distancia se ajusta bastante bien en la realidad, por lo que se podrían usar este tipo de simulaciones sin problemas en los entornos locales a los que están dirigidos.

La posibilidad de obtener el comportamiento de las redes WiFi en entornos locales, ampliamente usadas, abre la puerta a multitud de aplicaciones, desde la planificación de coberturas en entornos con muchos obstáculos hasta su uso para labores de posicionamiento local.

El hecho de poder automatizar la obtención de estos datos facilita estas labores enormemente, eliminando la necesidad de tomar medidas en la zona de interés.
Por ello, este método es más económico y rápido, además de permitir la obtención de datos desde prácticamente cualquier punto del entorno a evaluar.

Un modelo de interacción de los rayos más complejo que tenga en cuenta otros fenómenos como el la difracción, el \textit{scattering} o incluso posibles absorciones de energía por parte de materiales podrían arrojar resultados más ajustados a la realidad, pero implicaría una mayor complejidad y un mayor coste computacional.