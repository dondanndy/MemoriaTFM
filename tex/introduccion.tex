% Las redes inalámbricas de WiFi se han popularizado en la última década hasta el punto de ser un protocolo barato de usar más a allá de su uso original, con hardware compatible en multitud de dispositivos.

% La potencia de su señal, crítica en su uso principal y en otros como el posicionamiento local, es difícil de predecir en escenarios con obstáculos, de forma que aplicaciones que no los tengan en cuenta no podrán garantizar un desempeño correcto.

% Para poder conocer esta potencia en las zonas de interés surge la idea de usar la aproximación de las ondas a un rayo que se deplaza con el frente de onda, de modo que es posible conocer su recorrido teniendo en cuenta los posibles rebotes que se puedan producir.

% Esta técnica ha sido tradicionalmente usada en el desarrollo y posicionamiento de antenas de telefonía, principalmente para su uso en ciudades.
% En estos escenarios, los edificios harán las veces de obstáculos de modo que en ciertas zonas la señal puede ser más débil.

% La simulaciones con trazado de rayos buscan encontrar este tipo de zonas antes de su implantación, de modo que sea posible corregir las posibles deficiencias y conseguir un posicionamiento óptimo de las antenas usadas.

% En este trabajo se plantea su uso en un escenario local.
% Como se comentaba al inicio, la señal de WiFi encontrará en los escenarios donde se usa obstáculos en forma de paredes o grandes objetos que impidan su avance.

% Esto generará efectos de propagación multicamino, donde los rebotes de la señal en las paredes y obstáculos que se encuentren en su camino harán que en ciertos puntos lleguen, además de la señal original, una cierta potencia adicional o bien, en escenarios sin visión directa, la señal recibida será exclusivamente la de estos rebotes.

% De forma general, la modelización de estos efectos es imposible basándose únicamente en argumentos geométricos o estadísticos, por lo que será necesario simular la propagación de la señal con un mapa virtual del entorno a estudiar. 

Las redes inalámbricas de WiFi se han popularizado en la última década hasta el punto de ser un protocolo barato de usar más a allá de su uso original, con hardware compatible en multitud de dispositivos.

Su diseño solo contempla su uso en entornos interiores, donde existirán infinitas combinaciones de configuraciones: distintos tamaños de salas, distintas posiciones de mobiliario u objetos y con infinidad de materiales distintos.

Para poder conocer con seguridad el desepeño de estas ondas es necesario conocer cómo se propagan en este tipo de escenarios, tarea que puede llegar a ser imposible con técnicas analíticas o estadísticas.

Para poder conocer las características de la radiación en las zonas de interés surge la idea de usar la aproximación de las ondas a un rayo que se deplaza con el frente de onda, de modo que es posible conocer su recorrido teniendo en cuenta los posibles rebotes que se puedan producir. 

Esta técnica ha sido tradicionalmente usada en otros campos de la física como la óptica o en aplicaciones acústicas.
En el campo de las ondas de radio su aplicación se ha aplicado principalmente para el desarrollo y posicionamiento de antenas de telefonía, principalmente para su uso en ciudades. En estos escenarios, los edificios harán las veces de obstáculos de modo que en ciertas zonas la señal puede ser más débil. 

La simulaciones con trazado de rayos buscan encontrar este tipo de zonas antes de su implantación, de modo que sea posible corregir las posibles deficiencias y conseguir un posi- cionamiento óptimo de las antenas usadas.

En este trabajo se plantea su uso en un escenario local, en interiores. Como se comentaba al inicio, la señal de WiFi encontrará en los entornos donde se usa obstáculos en forma de paredes o grandes objetos que impidan su avance.

Esto generará efectos de propagación multicamino, donde los rebotes de la señal en las paredes y obstáculos que se encuentren en su camino harán que en ciertos puntos lleguen, además de la señal original, una cierta potencia adicional fruto de estos rebotes.
También encontraremos casos sin visión directa entre emisor y receptor, donde la señal recibida será exclusivamente la de estos rebotes.

La mayor ventaja de esta aproximación respecto a otros métodos de resolución de la ecuación de ondas es la posibilidad de paralelización.
Si se consideran los rayos independientes entre sí --es decir, se obvian los posibles efectos de interferencias-- se obtiene el caso idóneo de paralelización de código, donde no se necesitan barreras ni niguna otra técnica de programación paralela que pueda ralentizar la ejecución.

Por ello, en este trabajo se plantea el uso de GPUs --\textit{Graphic Processing Unit}--.
Este tipo de dispositivos cuentan con un gran número de núcleos ya que estan planteados para tareas muy paralelizables, por lo que son idóneas para este caso.

La programación en ese caso toma ciertas diferencias respecto a la programción para una CPU.
Existen diferencias en la manera de tratar la ejecución y la memoria que requerirán un cierto cuidado a la hora de pasar de un dispositivo a otro.

De forma concreta, los objetivos de este Trabajo de Fin de Máster serán
\begin{itemize}
    \item Diseñar e implementar un simulador de emisión de rayos que contemple su trayectoria en un entorno local.
    \item Implementar y estudiar la aceleración de la ejecución de la simulación en GPUs.
    \item Obtener datos en un entorno real con la ayuda de un robot autónomo y compararlos con los valores obtenidos en la simulación.
\end{itemize}

Con estos objetivos se busca aplicar las competencias adquiridos en el Máster en Simulación en Ciencias e Ingeniería en el ámbito de simulación de fenómenos naturales aplicando los conocimentos adquiridos sobre programación, especialmente la programación en paralelo.

