Las redes inalámbricas de WiFi se han popularizado en la última década hasta el punto de ser un protocolo barato de usar más a allá de su uso original, con hardware compatible en multitud de dispositivos.

La potencia de su señal, crítica en su uso principal y en otros como el posicionamiento local, es difícil de predecir en escenarios con obstáculos, de forma que aplicaciones que no los tengan en cuenta no podrán garantizar un desempeño correcto.

Para poder conocer esta potencia en las zonas de interés surge la idea de usar la aproximación de las ondas a un rayo que se deplaza con el frente de onda, de modo que es posible conocer su recorrido teniendo en cuenta los posibles rebotes que se puedan producir.

Esta técnica ha sido tradicionalmente usada en el desarrollo y posicionamiento de antenas de telefonía, principalmente para su uso en ciudades.
En estos escenarios, los edificios harán las veces de obstáculos de modo que en ciertas zonas la señal puede ser más débil.

La simulaciones con trazado de rayos buscan encontrar este tipo de zonas antes de su implantación, de modo que sea posible corregir las posibles deficiencias y conseguir un posicionamiento óptimo de las antenas usadas.

En este trabajo se plantea su uso en un escenario local.
Como se comentaba al inicio, la señal de WiFi encontrará en los escenarios donde se usa obstáculos en forma de paredes o grandes objetos que impidan su avance.
